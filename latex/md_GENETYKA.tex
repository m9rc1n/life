Witam w pierwszym wpisie z serii \char`\"{}\-D\-N\-A\char`\"{}. Poświęcę go liniom papilarnym. Pierwsze pytanie to czym one są. Dobra, widzimy jakieś wzorki, jednak nie jest to specjalnie dobre wyjaśnienie. Inna nazwa lini papilarnych to linie brodawkowe. Ta nazwa mówi już coś więcej. Jeszcze dokładniej można o nich powiedzieć, że są to wytwory skóry właściwej, bruzdy do których wnikają zakończenia nerwów. Mają je tylko ssaki naczelne i misie Koala.

Jak zapewne wiecie, dla każdego z nas są one niepowtarzalne. Nie ma możliwości by były dwie osoby z identycznym układem linii. Oczywiście taką właściwość człowiek wykorzystał. Zauważył ją w X\-I\-X wieku, jednak dopiero w latach 30. X\-X wieku zaczął ją wykorzystywać. Obecnie odciski palców są powszechnie używane przez policję i inne organizacje dbające o bezpieczeństwo. Dzięki nim (nierozważni) przestępcy mogą zacząć oglądać niebo w kratkę. Do zidentyfikowania sprawcy używa się tzw. minucji -\/ charakterystycznych cech jak np. rozwidleń, zagięć, haczyków itp. By mieć praktycznie 100\% pewność wystarczy już 12 identycznych minucji, jednak może wystarczyć mniej -\/ zależy to od ich rzadkości. Tyle też wystarczy by uznać odcisk za wiarygodny w świetle polskiego prawa. Aha. Nauka zajmująca się tymi sprawami to daktyloskopia.

Czasami można usłyszeć o zasadzie 3\-N Francisa Galtona. Mówi ona że linie są niepowtarzalne, niezmienne i nieusuwalne. O ile do pierwszych dwóch stwierdzeń nie można się przyczepić, to \char`\"{}nieusuwalne\char`\"{} nie jest już dobrym określeniem. W dzisiejszych czasach jeżeli tylko chcesz możesz się ich pozbyć. Oczywiście wiąże się to z zabiegiem dermatologicznym. Niestety zdarzają się sytuacje w których to natura podejmuje za nas decyzje. Dwie, bardzo rzadkie, choroby genetyczne wywołują brak linii papilarnych. Jest to spowodowane uszkodzeniem białka keratyna 14.

Na początku napisałem że wszyscy mają różne linie. Tutaj chciałbym wspomnieć, że nawet bliźnięta nie są wyjątkami. Mimo że mają takie samo D\-N\-A, ich linie papilarne są różne. Jednak o kwasie dezoksyrybonukleinowym napiszę za niedługo.

To tyle jeśli chodzi o linie papilarne. Zapraszam do następnego wpisu z serii \char`\"{}\-D\-N\-A\char`\"{}. Napiszę w nim o cechach dziedzicznych u człowieka -\/ już niebawem.

\subsection*{D\-N\-A\-: cechy dziedziczne}

Witam w drugim wpisie z serii \char`\"{}\-D\-N\-A\char`\"{}. Dziś napiszę o cechach dziedzicznych u człowieka. Jeżeli ktoś uważnie ogląda Dr. House'a, to w jednym odcinku chłopak stwierdził, że wie iż jest adoptowany już od 5 klasy -\/ znalazł w internecie informację, że podbródek który on ma, jest cechą dziedziczną, a żaden z jego przybranych rodziców go nie miał. Oczywiście nie chodzi mi tu abyśmy wszyscy zaczęli porównywać swoje cechy wyglądu z rodzicami. Powiedzmy że to co tutaj napiszę jest swoistą ciekawostką. Na dodatek niektóre cechy można odziedziczyć po przodkach (babci, dziadku, prababci).

Przejdźmy do konkretów. Dziedziczymy cechy fizyczne i psychiczne oraz biohemiczne. W psychicznych zawierają się m.\-in. podatności na pewne czynniki zewnętrzne. W fizycznych i biochemicznych dziedziczymy wygląd oraz cechy "wewnętrne (m.\-in. grupę krwi). Oto niektóre cechy dziedziczne\-:


\begin{DoxyItemize}
\item kolor oczu
\item kształt i wielkość oczu
\item ułożenie oczu
\item długość rzęs ( długie \char`\"{}dominują\char`\"{} nad krótkimi )
\item kształt uszu
\item kolor włosów
\item grupy krwi
\item czynnik Rh
\item inteligencja ( dziedziczna genowo i w dużym stopniu uwarunkowana czynnikami zewnętrznymi,środowiska )
\item leworęczność i praworęczność
\item umiejętność zwijania języka
\item sposób ułożenia dłoni \char`\"{}do pacierza\char`\"{} ( chodzi o to czy kciuk lewej ręki leży pod kciukiem prawej ręki czy na odwrót) Niestety dziedziczne są też choroby genetyczne.
\end{DoxyItemize}

W tym temacie warto wspomnieć o eugenice. Jest to nauka zajmująca się sterowaniem genami. Chodzi o to by stworzyć jak najlepszy organizm poprzez krzyżowanie osobników z porządanymi cechami. W ten sposób można otrzymać coraz to \char`\"{}lepszy\char`\"{} osobnik danego gatunku. Jeżeli chcecie to możecie zrobić taki eksperyment w domu. Wystarczy że macie akwarium i gupiki. Przyjmujemy że mamy kilka samców. Chodzi nam by uzyskać samca z jak największym (i najpiękniejszym) ogonem. Wystarczy go skrzyżować z samicą gupika, a następnie z potomstwa wybrać kolejnego samca z największym ogonem i tak dalej.

D\-N\-A\-: gust i indywidualizm

Podobno o gustach się nie rozmawia. Jednak czym on jest? Dodatkowo indywidualność. Dlaczego jesteśmy tacy jacy jesteśmy?

Gdy mówimy że ktoś ma dobry gust, chodzi nam jego wybory w kwestii estetycznej. Mówimy że ktoś urządził swoje mieszkanie gustownie -\/ czyli estetycznie i ze smakiem. Nie jest to cecha z którą się rodzimy, nie dziedziczymy jej. Gust musi zostać nam wpojony. Oczywiście wiele zależy od środowiska w którym żyje człowiek. Gustem można nazwać ocenianie wyżej pewnych wartości, a niżej innych. Oczywiście jest to poniekąd sprawa podświadoma i często polega na tym że wg. nas \char`\"{}coś pasuje, a coś nie\char`\"{}.

Drugi temat to indywidualność. Nasz własny styl. Podobnie jak gust, nie rodzimy się z nim. Kształtujemy go sobie sami. Duży udział w jego kształtowaniu ma środowisko w którym żyjemy, jego poglądy i sposób życia. Oczywiście indywidualność to także nasz stosunek do otoczenia. To czy jesteśmy \char`\"{}dowódcami\char`\"{}, czy może \char`\"{}robotnikami\char`\"{}. Jak odnajdujemy się i zachowujemy w naszym środowisku. Czy potrafimy mieć własne zdanie czy może idziemy razem z tłumem.

D\-N\-A\-: stres

Jak wiemy stres jest w naszym życiu obecny bez przerwy. Rano, kiedy idziemy do szkoły, podczas nauki, popołudniu kiedy gonią nas terminy, wieczorem kiedy to odrabiamy zadania domowe. Oczywiście istnieje świat bez szkoły, ale i on nie jest wolny od stresu -\/ praca, brak pracy (bezrobocie), obowiązki domowe itd. itp.

Jednakże czym on jest. Można powiedzieć że stres, jest to nasza (naszego organizmu) reakcja na czynniki zewnętrzne. Oczywiście czynniki te są w naszym odczuciu negatywne i przynoszą nieporządane skutki.

Oczywiście stres ma także pozytywne aspekty. Niekiedy potrafi motywować. Jednakże gdy jest on zbyt duży powoduje biegunowo odmienne zachowanie. Gdy jest on przewlekły (np. bardzo stresująca praca) może doprowadzić do różnego rodzaju schorzeń jak np. nerwica, lub odpowiedzi fizycznej organizmu (bóle, wzrost ciśnienia itd.).

Oczywiście stres można próbować opanować. Nie będę pisał jak, gdyż nie jestem jakimś specjalistą w tej dziedzinie. Jednak ważne jest uzmysłowienie sobie że mamy nad nim kontrolę. Niektórzy ludzie uważają że stres jest już odrębnym bytem będącym nieodłączną częścią ich życia. To nie jest dobre. Ludzie tacy obwiniają za taki stan rzeczy swoją osobowość, czynniki zewnętrzne jak np. pracę, znajomego, sytuację życiową. Mimo że może się wydawać że jest to jakaś metoda, nie jest to dobre. Ciągłe życie w stresie, jak pisałem na początku, jest niezdrowe, a umiejętność radzenia sobie z nim jest bardzo ważna.

Oczywiścią są pewne uwarunkowania dzięki którym jedni lepiej radzą sobie ze stresem, a inni gorzej. Zależy to, w głównej mierze, od środowiska w jakim żyjemy, ale także po trochu od genów.

Był to ostatni wpis z serii \char`\"{}\-D\-N\-A\char`\"{}. Dzięki wszystkim którzy to czytali. Piszcie opinie jak wam się podobała ta seria. 