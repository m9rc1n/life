\hypertarget{client_8hpp}{\subsection{client.\-hpp}
\label{client_8hpp}\index{src/client/client.\-hpp@{src/client/client.\-hpp}}
}

\begin{DoxyCode}
00001 \textcolor{preprocessor}{#include "../common/Map.hpp"}
00002 \textcolor{preprocessor}{#include "../common/MapObject.hpp"}
00003 \textcolor{preprocessor}{#include "../common/Creature.hpp"}
00004 \textcolor{preprocessor}{#include "../common/Herbivore.hpp"}
00005 \textcolor{preprocessor}{#include "../common/Predator.hpp"}
00006 \textcolor{preprocessor}{#include "../common/Config.hpp"}
00007 
00008 \textcolor{preprocessor}{#include <iostream>}
00009 \textcolor{preprocessor}{#include <mutex>}
00010 \textcolor{preprocessor}{#include <thread>} \textcolor{comment}{// c++11}
00011 \textcolor{preprocessor}{#include <chrono>} \textcolor{comment}{// c++11}
00012 
00013 \textcolor{keyword}{namespace }client
00014 \{
00015     \textcolor{keywordtype}{void} run(std::mutex *mutex, \hyperlink{classMap}{Map} *map, \hyperlink{structConfig}{Config} *config);
00016     \textcolor{comment}{// tutaj deklaracje jakichs klas uzytych w kliencie, np.}
00017     \textcolor{comment}{// class GameWindow;}
00018 \};
\end{DoxyCode}
